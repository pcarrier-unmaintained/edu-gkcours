\documentclass{beamer}
\usetheme{Copenhagen}

\usepackage[utf8]{inputenc}
\usepackage[francais]{babel}
\usepackage{hyperref}
\usepackage{textcomp}
\usepackage{xcolor}
\usepackage{graphicx}

\AtBeginSection[]
{
frametitle{Plan}
begin{frame}
\tableofcontents[currentsection,hideothersubsections]
end{frame}
}

\title{Programmation Neuro-linguistique}
\author{Geoffroy Carrier, Jean-Christophe Saad-Dupuy}
\institute{Université Joseph Fourier, L3 MIAGE}

\date{23 février 2009}

\begin{document}

\frame{\titlepage}

\section{Présentation historique}
\subsection{Émergence}
\subsection{Croissance}
\subsection{Récupération}

\section{Théorie}
\subsection{Fondements}
\subsection{Techniques}
\section{Dans l'environnement professionnel}
\subsection{Un outil pour comprendre}
\subsection{Un outil pour être compris}
\section{Limites et précautions}

\begin{frame}
	\begin{itemize}
	\item Présentation historique
		\begin{itemize}
			\item Emergeance
			\item Bases
			\item Vulgarisation / Recuperation
		\end{itemize}
	\item Théorie
	\item Dans l'environnement Professionnel
	\item Limites et précautions
	\end{itemize}
\end{frame}

\frame{
\frametitle{Exemple de frame}
Et c'est facile.
}

\end{document}
