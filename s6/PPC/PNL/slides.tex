\documentclass{beamer}
\usetheme{Copenhagen}

\usepackage[utf8]{inputenc}
\usepackage[francais]{babel}
\usepackage{hyperref}
\usepackage{textcomp}
\usepackage{xcolor}
\usepackage{graphicx}

\AtBeginSection[]
{
\begin{frame}
\frametitle{Plan}
\tableofcontents[currentsection,hideothersubsections]
\end{frame}
}

\title{Programmation Neuro-linguistique}
\author{Geoffroy Carrier, Jean-Christophe Saad-Dupuy}
\institute{Université Joseph Fourier, L3 MIAGE}

\date{23 février 2009}

\begin{document}

\frame{\titlepage}

\begin{frame}
\frametitle{Plan}
\tableofcontents
\end{frame}

\section{Présentation historique}
\subsection{Émergence}
\frame{
\frametitle{PNL}
\begin{itemize}
	\item fondée dans les années 70
	\item par John Grinder et Richard Bandler
\end{itemize}
}

\frame{
\frametitle{Bases de la PNL}
Repose sur les travaux de 3 psychiatres:
\begin{itemize}
	\item Fritz Perls (Gestalt Therapy)
	\item Virginia Satir (Thérapie familiale)
	\item Milton Erickson (Hypnose Ericksonienne)
\end{itemize}
}

\frame{
\frametitle{Gestalt Therapy}
Objectif :
\begin{itemize}
	\item Augmenter la capacité d'adaptation à des êtres ou des environnements différents
\end{itemize}
}

\frame{
\frametitle{Gestalt Therapy}
	Intègre cinq dimensions principales :
	\begin{itemize}
		\item  physique
		\item  affective
                \item cognitive
                \item sociale
                \item spirituelle		
	\end{itemize}
}

\frame{
\frametitle{La thérapie familiale}
	\begin{itemize}
		\item Les troubles d'un individu sont symptome du dysfonctionnement dans lequel ce dernier évolue.
		\item Thérapie intervenant sur tous les membres du groupe.
	\end{itemize}
}
\frame{
\frametitle{Hypnose Ericksonienne}
	\begin{itemize}
		\item L'inconscient est profondement bon et puissant
		\item Méthode de communication avec l'inconscient
	\end{itemize}
}
\frame{
\frametitle{Hypnose Ericksonienne}
	Champs d'applications :
	\begin{itemize}
		\item états depressifs
		\item phobies, Angoisses
		\item TOC
		\item dépendances
		\item travail sur le deuil
		\item ...
	\end{itemize}
}

\frame{
\frametitle{Hypnose Ericksonienne}
	Objectif : mobiliser le conscient et l'inconscient pour déclencher les 
	changements utiles à la résolution d'une problématique.
}

\subsection{Croissance}
\subsection{Récupération}

\section{Théorie}
\subsection{Fondements}
\subsection{Techniques}

\section{Dans l'environnement professionnel}
\subsection{Un outil pour comprendre}
\subsection{Un outil pour être compris}

\section{Limites et précautions}

\end{document}
