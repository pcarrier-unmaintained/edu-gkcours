\documentclass{beamer}
\usetheme[hideothersubsections,width=2cm]{Berkeley}

\usepackage[utf8]{inputenc}
\usepackage[francais]{babel}
\usepackage{hyperref}
\usepackage{textcomp}
\usepackage{xcolor}
\usepackage{graphicx}

\title{Programmation Neuro-Linguistique}
\author{Geoffroy Carrier,\\ J-Ch. Saad-Dupuy}
\institute{Université Joseph Fourier, L3 MIAGE}

\date{23 février 2009}

\begin{document}

\frame{\titlepage}

\frame{
\frametitle{Plan}
\tableofcontents
}

\section*{Introduction}
\frame{
\frametitle{Kézako ?}
Il existe énormément de définitions plus ou moins compatibles...
\only<1>{
	\begin{block}{Selon A}
	La la la
	\end{block}
}
\only<2>{
	\begin{block}<2>{Selon B}
	Bi bi bi
	\end{block}
}
}

\frame{
\frametitle{Objectifs}
\begin{itemize}
\item Modèles de comportements ;
\item Techniques pour augmenter les habiletés relationnelles ;
\end{itemize}
}

\section{Présentation historique}
\subsection{Émergence}
\frame{
\frametitle{PNL}
\begin{itemize}
	\item<+-> Fondée dans les années 70 ;
	\item<+-> Fohn Grinder et Richard Bandler.
\end{itemize}
}
\frame{
\frametitle{Bases de la PNL}
Repose sur les travaux de 3 psychiatres :
\begin{itemize}
	\item<+-> Fritz Perls (Gestalt Therapy)
	\item<+-> Virginia Satir (Thérapie familiale)
	\item<+-> Milton Erickson (Hypnose Ericksonienne)
\end{itemize}
}

\frame{
\frametitle{Gestalt Therapy}
Objectif :
\begin{itemize}
	\item<+-> Augmenter la capacité d'adaptation à des êtres ou des environnements différents ;
	\item<+-> restaurer la liberté de choix de l'individu.
\end{itemize}
}

\frame{
\frametitle{Gestalt Therapy}
	Intègre cinq dimensions principales :
	\begin{itemize}
		\item Physique ;
		\item Affective :
                \item Cognitive ;
                \item Sociale ;
                \item Spirituelle.
	\end{itemize}
}

\frame{
\frametitle{La thérapie familiale}
	\begin{itemize}
		\item<+-> Les troubles d'un individu sont symptômes du dysfonctionnement dans lequel ce dernier évolue.
		\item<+-> Thérapie intervenant sur tous les membres du groupe.
	\end{itemize}
}
\frame{
\frametitle{Hypnose Ericksonienne}
	\begin{itemize}
		\item L'inconscient est profondement bon et puissant
		\item Méthode de communication avec l'inconscient
	\end{itemize}
}
\frame{
\frametitle{Hypnose Ericksonienne}
	Champs d'applications :
	\begin{itemize}
		\item états depressifs
		\item phobies, Angoisses
		\item TOC
		\item dépendances
		\item travail sur le deuil
		\item ...
	\end{itemize}
}

\frame{
\frametitle{Hypnose Ericksonienne}
	Objectif : mobiliser le conscient et l'inconscient pour déclencher les 
	changements utiles à la résolution d'une problématique.
}

\subsection{Croissance}
\subsection{Récupération}
\frame{
\begin{itemize}\frametitle{Récupération}
	\item<+-> Bibliographie abondante ;
	\item<+-> Enormément repris par des centres de formations.
\end{itemize}
}
\section{Théorie}
\subsection{Fondements}
\begin{frame}[t]\frametitle{Experts en communication}
	\begin{itemize}
		\item Une acuité sensorielle développée ;
		\item Une capacité à établir le rapport ;
		\item Un respect réel du modèle du monde de l'autre ;
		\item Un art de poser des questions précises ;
		\item Beaucoup de flexibilité et d'adaptabilité ;
		\item Une adaptation facile ;
		\item Une aptitude à établir et poursuivre des objectifs spécifiques.
	\end{itemize}
\end{frame}
\subsection{Techniques}

\section{Dans l'environnement professionnel}
\subsection{Un outil pour comprendre}
\subsection{Un outil pour être compris}

\section{Limites et précautions}
\begin{frame}[t]\frametitle{La MILS interpelle le législateur}
	Rapport MILS 2001 :
	\only<1>{
		\begin{quote}
		La programmation neurolinguistique, couramment dénommée PNL,
		forme un ensemble disparate de méthodes de communication
		(apprendre à reformuler un message, à décoder des signaux non verbaux,
		des mouvements oculaires, etc..),
		basé sur un ensemble tout aussi disparate de références théoriques.
		Les fondements scientifiques et les validations empiriques sont faibles~:
		les hypothèses relatives aux mouvements oculaires ont d'ailleurs été infirmées.
		\end{quote}
	}
	\only<2>{
		\begin{quote}
		L'absence de principes déontologiques orientés vers l'aide et la santé,
		plutôt que vers l'exploitation et le profit,
		l'absence de connaissances en psychopathologie et en psychiatrie
		permettant d'aider ou d'orienter les personnes perturbées,
		l'absence de formation scientifique permettant de relativiser les connaissances
		et de ne pas prétendre à la vérité, caractérisent les pratiques qui font question.
		\end{quote}
	}
\end{frame}

\frame{\frametitle{}
	\begin{itemize}
		\item<+-> En tant qu'outil, la PNL peut être se réveler enrichissante ;
		\item<+-> En entreprise attention aux dérives :
			\begin{itemize}
				\item<+-> au niveau du recrutement
			\end{itemize}
	\end{itemize}
}
\end{document}
